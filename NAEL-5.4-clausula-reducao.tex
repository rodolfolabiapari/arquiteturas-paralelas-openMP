\definecolor{bostonuniversityred}{rgb}{0.8, 0.0, 0.0}
\subsection{Cláusula de Redução}

\begin{frame}[fragile]{OpenMP - Cláusula de Redução}
	\begin{itemize}
		\item Suponhamos que temos a seguinte variável:
	\end{itemize}
\begin{lstlisting}
...
double resultado_global = 0.0;
...
\end{lstlisting}
\pause
	\begin{itemize}
		\item Agora, suponhamos que temos a seguinte função, que calcula algum resultado que será armazenado em (\texttt{resultado\_global}) e desejamos paralelizá-la com a diretiva \texttt{parallel}:
	\end{itemize}
\begin{lstlisting}
...
calcula_resultado();
...
\end{lstlisting}
\end{frame}

\begin{frame}[fragile]
	\begin{itemize}
		\item Essa função pode ser implementada de duas maneiras distintas:
	\end{itemize}
\fontsize{7.5pt}{7.2}\selectfont
\begin{lstlisting}[caption=Implementação 1]
double resultado_global = 0.0;
...
void calcula_resultado(double *resultado_global);
...
# pragma omp parallel num_threads(thread_count)
calcula_resultado(&resultado_global)
\end{lstlisting}
\pause
\begin{lstlisting}[caption=Implementação 2, label=impl2]
double resultado_global = 0.0;
double calcula_resultado();
# pragma omp parallel num_threads(thread_count)
{
	# pragma omp critical
	resultado_global += calcula_resultado();
}
\end{lstlisting}
\pause
\fontsize{10pt}{7.2}\selectfont
\begin{itemize}
	\item \textbf{{\color{bostonuniversityred} Alguém vê algum problema com a implementação 2?}}
\end{itemize}

\end{frame}

\begin{frame}[fragile]{OpenMP - Cláusula de Redução}
\begin{itemize}
	\item Devido à sessão crítica, cada chamada de \texttt{calcula\_resultado} pode ser invocada somente por uma \textit{thread} por vez, ou seja, \textbf{sequencialmente}.
	\medskip
	\pause
	\item Uma possível solução:
\end{itemize}
\fontsize{8pt}{7.2}\selectfont
\begin{lstlisting}
...
double resultado_global = 0.0;
...
double calcula_resultado();
...
# pragma omp parallel num_threads(thread_count)
{
	double resultado_privado = 0.0;
	resultado_privado += calcula_resultado();
	# pragma omp critical
	resultado_global += resultado_privado;
}
\end{lstlisting}
\fontsize{10pt}{7.2}\selectfont
\end{frame}

\begin{frame}[fragile]{OpenMP - Cláusula de Redução}
\begin{itemize}
	\item Essa é uma possível solução, mas a OpenMP provê uma solução mais limpa e elegante através da especificação de uma \textbf{variável de redução}.
	\smallskip
	\begin{itemize}
		\item Variável que interage com um \textbf{operador de redução}.
		\smallskip
		\begin{itemize}
			\item Operador binário (\textbf{Ex:} Adição e Multiplicação).
			\smallskip
			\item \textbf{Redução:} Cômputo que aplica o mesmo operador de redução à uma sequência de operandos para conseguir um único resultado.
		\end{itemize}
	\end{itemize}
	\medskip
	\pause
	\item Para tal, adiciona-se a cláusula \texttt{reduction} à diretiva \texttt{parallel}:
\end{itemize}
\fontsize{8pt}{7.2}\selectfont
\begin{lstlisting}
...
double resultado_global = 0.0;
...
double calcula_resultado();
...
# pragma omp parallel num_threads(thread_count) \
reduction(+: resultado_global)
{
	resultado_global += calcula_resultado();
}
\end{lstlisting}
\fontsize{10pt}{7.2}\selectfont
\end{frame}

\begin{frame}{OpenMP - Cláusula de Redução}
\begin{itemize}
	\item Internamente, o compilador vai criar uma variável privada para cada \textit{thread} dentro de uma seção crítica.
	\medskip
	\pause
	\item Nesse exemplo, todas as variáveis privadas são inicializadas em \texttt{0.0}.
	\medskip
	\pause
	\item Sintaxe da cláusula de redução:\\ \texttt{reduction(<operador>: <lista de variáveis>) }.
	\smallskip
	\begin{itemize}
		\item Operadores válidos:$ +, *, -, \&, |, $\textasciicircum$, \&\&, ||$.
	\end{itemize}
\end{itemize}
\end{frame}

\begin{frame}[fragile]{OpenMP - Cláusula de Redução}
\begin{itemize}
	\item Alguns problemas com a cláusula de redução:
	\medskip
	\begin{itemize}
		\item Operação de subtração não é associativa\footnote{Diferente comportamento em uma expressão com e sem parênteses.} ou comutativa, por exemplo:
	\end{itemize}
\fontsize{8pt}{7.2}\selectfont
\begin{lstlisting}
resultado = 0;
for (i = 1; i <= 4; i++)
	resultado -= i;
\end{lstlisting}
\fontsize{10pt}{7.2}\selectfont
	\begin{itemize}
		\item A versão sequencial desse código retorna $-10$.
		\medskip
		\item Separando as operações em duas \textit{threads}, onde a \textit{thread} 0 executa $(-1) + (-2)  =-3$ e a \textit{thread} 1 executa $(-3) + (-4)=-7$.
		\medskip
		\item O resultado final será $-3-(-7) =$ {\color{bostonuniversityred}$4$}.
		\medskip
		\item O correto seria: $-3+(-7) = -10$
	\end{itemize}	
\bigskip
\pause
\item A aritmética de ponto flutuante também não é associativa:\\ $((a + b) + c)$ e $(a +(b + c))$ podem não ter o mesmo resultado exato.
\end{itemize}
\end{frame}
