\section{Cláusula de Redução}

\frame{
    \Huge \color{blue} \bf  \centering Reduction
}

\begin{frame}[fragile]{Reduction - Introdução}
	\begin{lstlisting}
void Trap(double a, double b, int n, double* global_result_p);
	\end{lstlisting}

	\begin{itemize}
		\item Esta, provavelmente, não seria a chamada de função que utilizaríamos.

				\pause

		\item Utilizaríamos algo como:
	\end{itemize}

	\begin{lstlisting}
double Trap(double a, double b, int n);
	\end{lstlisting}

			\pause

	\begin{itemize}
		\item Assim, teríamos a função:
	\end{itemize}

	\begin{lstlisting}
global_result = 0.0;
# pragma omp parallel num_threads(thread_count) 
{
    #   pragma omp critical
    global_result += Trap(double a, double b, int n); 
}
	\end{lstlisting}

			\pause

	\begin{itemize}
		\item Mas há algo errado nesta função. O quê?
	\end{itemize}
\end{frame}




\begin{frame}[fragile]{Reduction - Introdução}
	\begin{lstlisting}
global_result = 0.0;
# pragma omp parallel num_threads(thread_count)
{
	double my_result = 0.0;
	my_result += Trap(double a, double b, int n); 
	#	pragma omp critical
	global_result += my_result;
}
	\end{lstlisting}

	\begin{enumerate}
		\item A {\bf chamada} está {\bf fora da seção crítica}.
		\item As threads poderão ser chamadas/executadas simultaneamente.
	\end{enumerate}
\end{frame}




\begin{frame}[fragile]{Reduction - Conceitos}
	\begin{itemize}
		\item OpenMP provê outras alternativas para esse tipo de situação.
		\item É denominado {\bf \textit{redução} da variável}.
		\begin{itemize}
			\item {\bf Reduction Operator:} Qualquer tipo de operador bináio em C/C++ (\verb!+, *, -, &, | , ^, &&, ||!);
			\item {\bf Reduction:} Computação que aplica repetidadamente o mesmo operador para uma sequência de operandos para pegar um simples resultado;
			\item {\bf Reduction Variable:} Variável de armazemamento (\verb&global_result&).
		\end{itemize}
	\end{itemize}

	\begin{lstlisting}
int sum = 0;
for (i = 0; i < n; i++)
	sum += A[i];
	\end{lstlisting}

	\begin{itemize}
		\item Em OpenMP, é possível realizar uma \textbf{redução} de uma específica variável ({\bf variável redutiva}) que utiliza um operador binário (\textbf{operador redutível}) por meio de diretivas.
	\end{itemize}
\end{frame}




\begin{frame}[fragile]{Reduction - Aplicação}
	\begin{lstlisting}
global_result = 0.0;
# pragma omp parallel num_threads(thread_count)
{
	double my_result = 0.0;
	my_result += Trap(double a, double b, int n); 
	#	pragma omp critical
	global_result += my_result;
}
	\end{lstlisting}

	\begin{itemize}
		\item Poderia ser reescrito como:
	\end{itemize}

	\begin{lstlisting}
global_result = 0;
# pragma omp parallel num_threads(thread_count) \
		reduction(+: global_result)
global_result += Trap(double a, double b, int n);
	\end{lstlisting}
\end{frame}




\begin{frame}[fragile]{Reduction - Aplicação}
	\begin{lstlisting}
global_result = 0;
# pragma omp parallel num_threads(thread_count) \
		reduction(+: global_result)
global_result += Trap(double a, double b, int n);
	\end{lstlisting}

			\bigskip

	\begin{enumerate}
		\item Cria-se uma variável privada para cada thread de uma team;
		\item Os dados de cada thread são gravados nestas;
		\item Também cria-se uma seção crítica para adição dessas com o valor final.
	\end{enumerate}

			\pause

	\begin{itemize}
		\item A Redução de uma variável \verb%float% ou \verb%double% pode resultar em valores ligeiramente diferentes.
		\begin{itemize}
			\item $a + (b + c)$ pode não ter o mesmo valor que usar $(a + b) + c$.
		\end{itemize}
		\item Quando uma variável é reduzida, automaticamente é compartilhada.
	\end{itemize}
\end{frame}




\begin{frame}[fragile]{Reduction - Aplicação}

	\begin{lstlisting}
global_result = 0;
# pragma omp parallel num_threads(thread_count) \
		reduction(+: global_result)
global_result += Trap(double a, double b, int n);
	\end{lstlisting}

	\begin{itemize}
		\item possui a mesma função que:
	\end{itemize}

	\begin{lstlisting}
global_result = 0.0;
# pragma omp parallel num_threads(thread_count)
{
	double my_result = 0.0;
	my_result += Trap(double a, double b, int n); 
	#	pragma omp critical
	global_result += my_result;
}
	\end{lstlisting}

	\begin{itemize}
		\item As variáveis privadas são inicializadas de acordo com o operando
				\pause
		\begin{itemize}
			\item Operando redutor \verb&+& inicializa as variáveis com o valor 0;
			\item Operando redutor \verb&*& inicializa as variáveis com o valor 1.
		\end{itemize}
	\end{itemize}
\end{frame}
